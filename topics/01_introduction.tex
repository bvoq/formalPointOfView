
\section{The formalist point of view}
Very concisely the formalist believes: "Everything is a formal language". \\
"The limits of my language mean the limits of my world." ~ Ludwig Wittgenstein \\
There are a lot of "formalist views" of the world/of mathematics/of art/of the mind, but before diving into these views I want to deliver a solid understanding of the concepts \textbf{formal language}, \textbf{formal systems}, \textbf{computation} and \textbf{formal logic}.

% Formal languages are what Wittgenstein would call pictures in the mind.

\iffalse
In the following section I will attempt to describe the theory of formal language, formal calculation and formal systems using the natural language English. There are various textbooks on this matter if you want to learn these concepts.

 

Now there are a lot of formalist views of the world and in this section some of these views are swiftly presented. 

Before presenting these however, we should have an understanding of what a formal language is, what a formal calculation (also referred to as a substitution) and what a formal system (also referred to as a Semi-Thue system) is. I will attempt to explain these concepts in the natural language English, however English is not a minimal formal language which is why this text might 

I'll also show a correspondence (a way of transforming one thing into the other) between formal languages, formal calculations and formal systems.

In a later section we will also lie the foundations of communication based on these two concepts and the various formalist philosophical beliefs.
  


Later we will try to define what communication is.


Knowledge is formal systems.



In the philosophy of mathematics the formalist believes that mathematics simply studies formal systems. Although mathematicians themselves use natural languages like English in their proof in principle everything that the mathematician calls a valid proof must be able to be turned into derivations in a formal system.


In physics (and all of science which uses the principle of verifiability)  (logical positivists) also work on the core assumption that the world acts formally (or at least sufficiently formal as though we can )


These are all interesting positions that can turn people the wrong way: Underlying 

If the universe were a probabilistic formal system
Why the probabilistic nature of 


All scientists (at least in the sense of logical positivism) have the core assumption, namely that our world acts in a formal way. If it wouldn't act in a formal way, the scientists couldn't create a model to describe the 
In the philosophy of physics the underlying assumption that our world is formal 



In the same vain some physicists believe that our physical world is also merely a formal system, a similar view that the pythagoreans shared for a long time.

However the formalist point of view states that everything that is knowable is formal. This text should point out an isomorphism between a formal system and a formal language and will then continue to derive

\fi